\documentclass{article}

% Esto es para poder escribir acentos directamente:

% Esto es para que el LaTeX sepa que el texto est� en espa�ol:
\usepackage[spanish]{babel}

% Paquetes de la AMS:
\usepackage{amsmath, amsthm, amsfonts}
%Margenes para el articulo
\oddsidemargin 0.3cm
\textwidth= 17cm
\textheight= 25.5cm
\headsep= 0.5cm
\hoffset= -1cm
\voffset= -2cm


%--------------------------------------------------------------------------
\title{ RUP v/s XP \\ Tarea 1 Ayudant\'ia}
\author{Fernanda Lemunguir Sep\'ulveda, Camilo Candia Rubinstein\\
  \small Universidad Tecnol\'ogica Metropolitana\\
  \small Ingenier\'ia de Software\\
  \date\small{18 de Octubre de 2013}
}

\begin{document}
\maketitle

\abstract{Una metodolog\'ia es un conjunto integrado de t\'ecnicas y m\'etodos que permite abordar de forma
homog\'enea y abierta cada una de las actividades del ciclo de vida de un proyecto de desarrollo. Es un proceso de software detallado y completo.

Una Metodolog\'ia para el desarrollo de software comprende los procesos a seguir sistem\'aticamente para idear, implementar y mantener un producto software desde que surgue la necesidad del producto hasta que se cumple el objetivo por el cual fue creado. }

\section{Metodolog\'ia RUP}

 Las siglas RUP en ingles significa Rational Unified Process (Proceso Unificado de Racional) es un producto del proceso de Ingenier\'ia de Software que proporciona un enfoque disciplinado para asignar tareas y responsabilidades dentro de una organizaci\'on del desarrollo. Su meta es asegurar la producci\'on del software de alta calidad que resuelve las necesidades de los usuarios dentro de un presupuesto y tiempo establecido.

\subsection{Ciclo de vida del proyecto: Fases}
RUP determina que el ciclo de vida del proyecto consiste en cuatro fases que permiten que el proceso sea presentado a alto nivel de una forma similar a un estilo de cascada. Cada fase tiene un objetivo clave y un hito al final que denota que el objetivo se ha logrado.

Las cuatro fases en las que divide el ciclo de vida del proyecto son:

1. Fase de Iniciaci\'on: se define el alcance del proyecto.

2. Fase de Elaboraci\'on: se analizan las necesidades en mayor detalle y se definen sus principios arquitect\'onicos.

3. Fase de Construcci\'on: se crea el dise�o y el c\'odigo fuente.

4. Fase de Transici\'on: se entrega el sistema a los usuarios.

Planear las 4 fases incluye: Asignaci\'on de tiempo, Hitos Principales, Iteraciones por Fases y Plan de proyecto.

\subsection{6 Principios Claves}

1. Adaptaci\'on del proceso: El proceso deber\'a adaptarse a las necesidades del cliente.

2. Balancear prioridades: Los requisitos de los diversos participantes pueden ser diferentes, contradictorios o disputarse recursos limitados. Debe encontrarse un equilibrio que satisfaga los deseos de todos. 

3. Colaboraci\'on entre equipos: Debe haber una comunicaci\'on fluida con el equipo para coordinar requisitos, desarrollo, evaluaciones, planes, resultados.

4. Demostrar valor iterativamente: Los proyectos se entregan en etapas iteradas. En cada iteraci\'on se analiza la opini\'on de los inversores, la estabilidad y calidad del producto.

5. Elevar el nivel de abstracci\'on.

6. Enfocarse en la calidad: El aseguramiento de la calidad forma parte del proceso de desarrollo y no de un grupo independiente.

\section{Metodolog\'ia XP}
La programaci\'on extrema es una metodolog\'ia de desarrollo \'agil basada en una serie de valores y de pr\'acticas de buenas maneras que persigue el objetivo de aumentar la productividad a la hora de desarrollar programas.
Este modelo de programaci\'on se basa en una serie de metodolog\'ias de desarrollo de software en la que se da prioridad a los trabajos que dan un resultado directo y que reducen la burocracia que hay alrededor de la programaci\'on.
Una de las caracter\'isticas principales de este m\'etodo de programaci\'on, es que sus ingredientes son conocidos desde el principio de la inform\'atica. El objetivo que se persegu\'ia en el momento de crear esta metodolog\'ia era la b\'usqueda de un m\'etodo que hiciera que los desarrollos fueran m\'as sencillos. Aplicando el sentido com\'un.

\subsection{Proceso de Desarrollo}
1. Interacci\'on con el cliente: En este tipo de programaci\'on el cliente pasa a ser parte implicada en el equipo de desarrollo. Su importancia es m\'axima en el momento de tratar con los usuarios y en efectuar las reuniones de planificaci\'on. Tiene un papel importante de interacci\'on con el equipo de programadores, sobre todo despu\'es de cada cambio, y de cada posible problema localizado, mostrando las prioridades. De esta forma se posibilita que el cliente pueda ir cambiando de opini\'on sobre la marcha, pero a cambio han de estar siempre disponibles para solucionar las dudas del equipo de desarrollo.

2. Planificaci\'on del Proyecto: En este punto se tendr\'a que elaborar la planificaci\'on por etapas, donde se aplicar\'an diferentes iteraciones. Para hacerlo ser\'a necesaria la existencia de reglas que se han de seguir por las partes implicadas en el proyecto para que todas las partes tengan voz y se sientan realmente part\'icipes de la decisi\'on tomada.
Las entregas se tienen que hacer cuanto antes mejor, y con cada iteraci\'on, el cliente ha de recibir una nueva versi\'on.

3. Dise\~no, Desarrollo y Pruebas: El desarrollo es la parte m\'as importante en el proceso. Todos los trabajos tienen como objetivo que se programen lo m\'as r\'apidamente posible, sin interrupciones y en direcci\'on correcta.
Tambi\'en es muy importante el dise\~no, y se establecen los mecanismos, para que \'este sea revisado y mejorado de manera continuada a lo largo del proyecto, seg\'un se van a\~nadiendo funcionalidades al mismo. Cada programador puede trabajar en cualquier parte del programa.De esta manera se evita que haya partes "propietarias de cada programador". 

\section{Cuadro comparativo: RUP v/s XP}

\begin{tabular}{|c|p{4cm}|p{4cm}|}
\hline
Caracter\'isticas & Metodolog\'ia RUP & Metodolog\'ia XP\\
\hline
Descripci\'on & Forma disciplinada de asignar tareas y responsabilidades en una empresa de desarrollo (qui\'en hace qu\'e, cu\'ando y c\'omo)& Nace en busca de simplificar el desarrollo del software y que se lograr\'a reducir el costo del proyecto\\
\hline
Metodolog\'ia de Procesos& An\'alisis y dise\~no.

Requerimiento.

Prueba Desarrollo.& Planificaci\'on, Dise\~no, Codificaci\'on, Pruebas, Especificaci\'on de casos de uso.\\
\hline
Metodolog\'ias de Datos & Casos de Uso.

Modelo Conceptual.

Diagrama de secuencias & Diagrama de clases.

Diagrama de casos de uso.

Tabla de requerimientos.

Historias de usuario.\\
\hline
Tama\~no de grupo & Requiere un grupo grande de programadores para trabajar con esta metodolog\'ia & Se requiere un grupo peque\~no de programadores para trabajar con esta metodolog\'ia entre 2 � 15 personas y estas ir\'an aumentando conforme sea necesario \\
\hline
Duraci\'on de proyectos & Largo & Corto\\
\hline
Detenci\'on de errores & En forma temprana & A largo plazo\\  
\hline
&RUP intenta reducir la complejidad del software por medio de estructura y la preparaci\'on de las tareas pendientes en funci\'on de los objetivos de la fase y actividad actual &  XP, como toda metodolog\'ia \'agil, lo intenta por medio de un trabajo orientado directamente al objetivo, basado en las relaciones Interpersonales y la velocidad de reacci\'on.\\
\hline
Roles & Analistas.

Desarrolladores.

Gestores.

Especialistas.

Revisor. & Programador.

Encargado de pruebas.

Cliente.

Encargado del seguimiento.

Consultor.

Gestor\\
\hline

\hline
\end{tabular}

%url repositorio

\section{Repositorio}

https://github.com/FernandaLemunguir/ayudantiaISW

% Bibliograf�a.
%-----------------------------------------------------------------
\begin{thebibliography}{99}

\bibitem http://www.usmp.edu.pe/publicaciones/boletin/fia/info49/articulos/RUP%20vs.%20XP.pdf
\bibitem http://mtdologiarup.blogspot.com/
\bibitem http://fabianbermeop.blogspot.com/2010/12/metodologia-rup-desarrollo-de-software.html
\bibitem http://tallerinf281.wikispaces.com/file/view/METODOLOG%C3%8DAS+TRADICIONALES.pdf

\end{thebibliography}

\end{document}